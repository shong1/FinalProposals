\documentclass[10pt]{article}
\usepackage{upgreek}
\usepackage{amsmath}
\usepackage{graphicx}
\usepackage{subfig}
\usepackage{hyperref}
\usepackage{mathrsfs}
\usepackage{amssymb}
\usepackage{caption}
\usepackage{stmaryrd}
\usepackage{cite}



\graphicspath{{/ReportImg}}
\DeclareGraphicsExtensions{.pdf,.png,.jpg,.tif}

%opening
\title{3D Computer Vision \\ Final Project Proposal}
\author{Sungmin Hong}
\date{\today}

\begin{document}

\maketitle

\section{Introduction}

The main theme of the final project is reconstructing 3D face from multiple sources. 
The first part of the project follows a framework suggested in the `Face reconstruction from the wild'~\cite{Kemel2011}.
The authors proposed the 3D face reconstruction method from large unstructured image collections, such as, images from google search or personal photo collections. 

\section{Algorithm Overview}

\noindent \textbf{Normalize Face Images \\} 

Since images from internet or personal collections are unstructured, we need to normalize each image to a canonical face, which is frontal view of a face.
In ~\cite{Kemel2011}, they used a fiducial-based approach to normalize faces. They first detect a face in the image by using the object detection method using the soft cascade method~\cite{Bourdev05}.
Fiducial points of a face, such as, corners of eyes, nostrils, tip of a nose, and etc. are extracted from the detected region by using the fiducial points detection method ~\cite{Everingham06}. 
The matlab codes for the face detection method is provided by the matlab package and the fiducial points detection method is provided in the matlab Computer Vision package. 

Let $q$ and $Q$ be the fiducials in 2D images and the 3D template shape. 
They assumed 2D points in each image are the weak perspective projection of 3D points on the template surface. 

\begin{equation}
    q = P_wQ = [ sR | t ]Q \\    
\label{eq1_1}
\end{equation}

To estimate $P_w$, they subtracted the centroid and calculate the pseudo-inverse of relative positions of 3D points.







\noindent \textbf{Initial Lighting and Shape Estimation \\}




\noindent \textbf{Surface Refinement using Local Shape Estimation \\}

\noindent \textbf{Ambiguity Recovery \\}

\noindent \textbf{Integration \\}

\noindent \textbf{Update Template Surface \\}



\bibliography{Proposal}{}
\bibliographystyle{plain}
\end{document}
